\documentclass[12pt, letterpaper, titlepage]{article}
\usepackage[left=3.5cm, right=2.5cm, top=2.5cm, bottom=2.5cm]{geometry}
\usepackage[MeX]{polski}
\usepackage[utf8]{inputenc}
\usepackage{graphicx}
\usepackage{enumerate}
\usepackage{amsmath} %pakiet matematyczny
\usepackage{amssymb} %pakiet dodatkowych symboli
\title{cwiczenie 3 }
\author{Kamil Przytulski}
\date{06.11.2022}
\begin{document}
\maketitle

\begin{center}
\begin{tabular}{ c c c }
cell1 & cell2 & cell3 \\
cell4 & cell5 & cell6 \\
cell7 & cell8 & cell9
\end{tabular}
\end{center}


\begin{table}[h]
\centering\caption{Tu wstawiamy opis Tabeli}
\begin{tabular}{c c c}
cell1 & cell2 & cell3\\
cell4 & cell5 & cell6\\
cell5 & cell6 & cell7
\end{tabular}
\end{table}



\begin{table}[h]
\centering\caption{Tu wstawiamy opis Tabeli}
\begin{tabular}{c|c|c}
cell1 & cell2 & cell3\\
cell4 & cell5 & cell6\\
cell5 & cell6 & cell7
\end{tabular}
\end{table}


\begin{table}[h]
\centering\caption{Tu wstawiamy opis Tabeli}
\begin{tabular}{c c c}
cell1 & cell2 & cell3\\
\hline
cell4 & cell5 & cell6\\
\hline
cell5 & cell6 & cell7\\
\hline
\end{tabular}
\end{table}

\begin{table}[h]
\centering\caption{Tu wstawiamy opis Tabeli}
\begin{tabular}{|l|l|l|}
\hline
cell1 & cell2 & cell3\\
\hline
cell4 & cell5 & cell6\\
\hline
cell5 & cell6 & cell7\\
\hline
\end{tabular}
\end{table}


\begin{table}[h]
\centering\caption{zad 1}
\begin{tabular} {c|ccc}
\hline
\hline
pacjent & Ból brzucha  &   temperatura ciała  & operacja \\
\hline
u1 & mocny   &   wysoka    &  tak\\
u2 & średni  &   wysoka    &  tak\\
u3 & mocny   &   srednia   &  tak\\
u4 & mocny   &   niska     &  tak\\
u5 & średni  &   srednia   &  tak\\
u6 & średni  &   srednia   &  nie\\
u7 & Mały    &   wysoka    &  nie\\
u8 & Mały    &   niska     &  nie\\
u9 & mocny   &   niska     &  nie\\
u10 & Mały   &   srednia   &  nie\\
\hline
\hline
\end{tabular} 
\end{table}



\end{document}

